
Two reflection tools have been implemented based on the LA infrastructure.
The time spent reflection tool presents the student with an overview of the
time spent percentages in the inquiry phases of an ILS (see Figure xxx1).
This tool is intended to be included in an ILS by the teacher (most often in
the Discussion phase).  The teacher can configure the tool by setting norm
percentages (shown as black bars in the figure).  Students can then reflect on
their personal time spent (blue bars) by answering reflection questions
configured by teachers.  The second reflection tool shows a time-line with the
ILS phases and when the student visited the phases (see Figure xxx2).  This
tool helps students reflect on the order in which they used the phases and for
how long.  Again, the teacher can define questions to prompt students to
initiate the reflection process.

Both (client-based) reflection tools query a reflection agent running on the
LA server.  This agent monitors incoming actions and collects the time spent
and phase transition actions for all students in all ILSs.  The two reflection
tools query the agent for the aggregated data for a particular student in an
ILS by providing the studentId and ilsId as parameters.  The agent sends the
aggregated data to the tool which creates the chart visualisation shown to the
student.
